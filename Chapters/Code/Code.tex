\documentclass[../../main.tex]{subfiles}

\begin{document}
\sloppy

\chapter{The code}

The main part of this thesis is the implementation of a self-consistent ladder-$\dga$ calculation with an emphasis on usability and minimal setup effort. The code is designed to be both accessible and performant, targeting fast execution through efficient parallelization and extensive memory optimizations. It is written in \texttt{Python} and relies mainly on \texttt{NumPy} for numerical computations, making use of vectorized operations and optimized linear algebra routines. In addition, the code carefully exploits symmetries inherent to the physical problem to further reduce computational cost and memory usage. In this chapter, we present the implementation of the computational framework developed to achieve a fully converged self-energy through the self-consistency algorithm of the ladder-$\dga$ equations and a subsequent calculation of the superconducting eigenvalues and gap functions for both singlet and triplet electron pairing. 
\\\\
The code is designed to efficiently compute vertex functions, such as the one-particle Green's function, irreducible and full vertices, three-leg vertices, the electron self-energy and many more, while maintaining flexibility for extensions and modifications. Building upon the underlying theoretical foundations which were established already in the previous chapters, this numerical implementation aims to translate the formal expressions into a working algorithm. Special attention has been given to structuring the code in a modular and efficient manner, ensuring that it can be used not only for the ladder-$\dga$ equations, but also for other algorithms that require multi-point vertex functions within the Matsubara frequency framework in condensed matter physics.
\\\\
The package will be provided under the MIT license, making it freely available for use, modification, and distribution. This permissive license promotes collaboration and allows academics and developers to make changes on the code without placing restrictions. By publishing the implementation open-source, we hope to encourage accessibility, reproducibility, and community development.
\\\\
In this chapter, we start with an overview of the overall structure and design choices of the code, followed by a detailed discussion of its core components and algorithms. We then describe the typical workflow for using the code, including input handling, computational steps and output generation. Finally, we discuss validation methods and potential improvements for future work.

\section{Structure and design choices}

Please note that there already exist two programs that allow the calculation of vertex functions and the self-energy through ladder-$\dga$: (i) \texttt{Abinitio$\dga$} \cite{Galler2019, Galler2018, Galler2017}, which is a one-shot multi-orbital calculation of the $\dga$ self-energy and (ii) \texttt{DGApy} \cite{Worm2023}, which is the one-band version of it, where additional quantities are calculated, such as the superconducting eigenvalues and real-valued Green's functions. The former is written in \texttt{Fortran}, whereas the latter has been developed in \texttt{Python}.
The purpose of the code developed in the course of this thesis is to provide the functionality of \texttt{Abinitio$\dga$}, while being easily accessible and quick to set-up, similar to \texttt{DGApy}. Therefore, we decided to write the program in \texttt{Python}. A main advantage of this programming language is its low entry-barrier and the easy set-up and code-execution. In theory, all that is needed is a \texttt{Python} environment with the necessary packages\footnote{For a detailed list of requirements including package versions, see the file \texttt{requirements.txt} in the source code directory.}, DMFT input files and the configuration file.

As already mentioned, there already exists a \texttt{Python} package, called \texttt{DGApy} \cite{Worm2023}, written by Paul Worm, which is capable of performing a single-band $\lambda$-corrected one-shot ladder-$\dga$ calculation. The code in this thesis will, however, be concerned with the multi-orbital implementation of the algorithm, additionally employing a self-consistency loop contrary to the $\lambda$-correction and will be capable of handling non-local Coulomb interactions. So far, \texttt{DGApy} has successfully been used in several instances, for example in Refs. \cite{DiCataldo2024, Worm2024}, to only mention a few. Since it is already widely used among researchers, the transition from \texttt{DGApy} to this code should be as simple as possible, hence the configuration file and some internal modules are designed very similar.

\subsection{Parallelization via MPI}

To lower the computation time, we employ the use of the \texttt{mpi4py} library in \texttt{Python}, which enables distributed computing across multiple processes in a straightforward and Pythonic manner. This allows the program to run inside a process pool, where objects can easily be sliced and passed to or gathered from other processes. While an ideal speed-up factor of $n$ for $n$ processes is theoretically possible, in practice, communication overhead and memory transfer costs typically limit the effective gain. Nevertheless, for large-scale computations involving high-dimensional vertex functions or large frequency boxes, this parallelization leads to significant reductions in runtime.

The central class handling parallelization is the \texttt{MpiDistributor}, which is initialized at the point where the explicitly non-local parts of the ladder-$\dga$ algorithm begin. This class abstracts the communication logic and provides a consistent interface for scattering and gathering objects across processes. The current implementation slices the data along the transferred momentum $\vv{q}$, which is possible due to the block-diagonal structure of most equations up to the Schwinger-Dyson equation (\ref{eq:sde_vrg}). Each MPI process is responsible for a specific set of $\vv{q}$-points, allowing for highly parallel and independent evaluation of vertex contractions.

Internally, the slicing is guided by knowledge of the irreducible Brillouin zone, and the distributor automatically determines the mapping from global $\vv{q}$-grid indices to process-local ones. This separation of concerns between parallelization logic and physics routines enhances maintainability and flexibility of the code. Further MPI optimizations, such as asynchronous communications or load balancing across uneven grid sizes, could be explored in future versions. 

\subsection{Memory reduction through symmetry}

The memory demands of a full ladder-$\dga$ calculation can be substantial, especially for multi-orbital models and large frequency ranges. To mitigate this, the code incorporates several symmetry-based optimizations that significantly reduce the memory footprint and enable efficient scaling to larger systems.

First, the calculation of non-local quantities is performed exclusively within the irreducible Brillouin zone. Thanks to the crystal symmetries of common lattice structures (e.g., square or quasi-1D lattices), only a small subset of the full momentum grid needs to be explicitly computed. The code determines this set via symmetry operations specified in the \texttt{brillouin\char`_zone} module, which ensures compatibility with arbitrary lattice symmetries defined in the configuration. For instance, on a $16\times16\times1$ grid, only 45 out of 256 $\vv{q}$-points are needed for a square lattice, which directly reduces both computational and memory load by more than a factor of five and even higher for larger grid sizes.

To further save memory, we keep most of the objects in half of their bosonic Matsubara frequency range: Due to the symmetry $F^{\omega\nu\nu'}_{\mathfrak{1234}}=(F^{(-\omega)(-\nu)(-\nu')}_{\mathfrak{4321}})^{*}=(F^{(-\omega)\nu\nu'}_{\mathfrak{4321}})^{*}$\footnote{The transformation from $-\nu, -\nu'$ to $\nu,\nu'$ is possible due to the symmetrized nature of the two-particle Green's function and the inherited vertex functions.}, we can restrict the objects to their positive bosonic Matsubara frequency region only. This saves half of the memory and thus a very large amount for the biggest objects, such as the auxiliary susceptibility or the full vertex.

These symmetry-based reductions are implemented in a transparent way for the user: the internal representation is compressed, but the high-level API provides access to the full data. This allows for both memory efficiency and usability, enabling the user to work with large systems that would otherwise be intractable on a single machine.

\subsection{Object-oriented API}

We tried to keep a very object-oriented approach when developing the toolbox for the program, as opposed to both \texttt{DGApy} and \texttt{AbinitioDGA}. Hence instead of directly working with \texttt{NumPy} arrays, we work with \texttt{(Local)FourPoint}, \texttt{(Local)Interaction}, \texttt{SelfEnergy}, \texttt{GreensFunction} and \texttt{GapFunction} classes. The reason being that in \texttt{Python}, it is very easy to overload operators, i.e. with so-called \say{\texttt{dunder}} methods, hence we implemented the most common operators \say{\texttt{+/-/@/$\sim$}}\footnote{Corresponding to addition, subtraction, multiplication and inversion.}, alongside multiplication and division with numbers, for these objects to help make the implementation of equations easier. This might not be an obvious improvement at first, however the underlying structure allows a very simple high-level handling of these objects, where most of the logic is performed in the background and only has to be coded once. We will discuss the explicit implementation of these \texttt{dunder} methods later.
\\\\
Compared to \texttt{DGApy}, we stripped down the module responsible for the Matsubara frequency handling, since most of the features now can be called directly from the objects themselves. Instead of calling a method from the \texttt{matsubara\char`_frequencies} module to cut an objects' frequency range, it is now a method on the object itself, e.g., \texttt{object.cut\char`_niv(10)}. Most of these methods are only implemented once in a base class which all subtypes inherit from. These base classes already cover most of the objects' functionality, where the subclass only implements features unique to the specific object type.
\\\\
All modules should be very self-explanatory, their names already provide a good description of what the module does and what its responsibilities are. For any details, we refer the reader to the documentation of the code inside the repository.
\\\\
One benefit of the object-oriented approach is the now easy implementation of various equations. For example, writing an arbitrary equation with arbitrary (non-local) vertex functions $A,B,C$ and $D$ like
\begin{align}
	D^{\q\nu\nu'}_{\mathfrak{1234}}=\frac{1}{2\beta} \sum_{\nu_1;\mathfrak{ab}} A^{\q\nu\nu_1}_{\mathfrak{12ab}}\left (B^{\omega\nu_1}_{\mathfrak{ba34}}+C^{\omega\nu_1\nu'}_{\mathfrak{ba34}}\right )
\end{align}
can be done with a call of \texttt{D = 1 / (2 * beta) * A.matmul(B.add(C))} or the \texttt{dunder} methods for matrix multiplication \say{\texttt{@}} and addition \say{\texttt{+}}, i.e., \texttt{D = 1 / (2 * beta) * A @ (B + C)}, where the index contraction, frequency handling and summation will be done in the background. This eases readability in the code and reduces redundancy. Furthermore, it allows for very straightforward optimization of contractions and multiplications through this architecture. Furthermore, inverting a vertex function is also very easy now, all that needs to be called is the \texttt{obj.invert()} function (or with a \say{\sim} prefix, i.e. \texttt{\sim obj}) on any object and the code performs the inversion in the background using \texttt{NumPy}'s vectorization and \texttt{SciPy}'s memory-optimized inversion by converting the object to compound indices. Furthermore, other operations, such as addition/subtraction in the form of \texttt{A.add(B)}/\texttt{A.sub(B)} or just simply \texttt{A\pm B} are possible and automatically handle different amounts of frequency dimensions, etc. There are a bunch of other convenience methods available, where we would like to refer the interested reader to the code and its documentation for more details.


\subsection{Setup and execution}

The code is very easy to set-up as it is available as an installable \texttt{Python} package, called \texttt{scdga}. After installing the necessary packages denoted in the \texttt{requirements.txt} file to the \texttt{Python} environment, one simply has to run \say{\texttt{pip install -e .}} from the repository directory in the terminal to install the package in editable mode. This allows the user to now access any module and class inside the \texttt{scdga} package. The main entry point to the program is the file \texttt{dga\char`_main.py} file, which can be started with either \say{\texttt{python~dga\char`_main.py}} for single-core execution (mostly used for testing purposes) or \mbox{\say{\texttt{mpiexec -np <n\char`_proc>~python~dga\char`_main.py}}} for multi-core processing. There are two additional command line parameters available to append to the execution:
\begin{itemize}
\item\say{\texttt{-p}}: With \say{\texttt{-p <path>}} one can specify the path to the configuration file, which contains all run-specific parameters. This is useful if one wants to store multiple configuration files in different directories. If this parameter is not set, the path defaults to the location of the repository directory.
\item\say{\texttt{-c}}: With \say{\texttt{-c <config name>}} one can specify the name of the configuration file one wants to load. This defaults to \texttt{dga\char`_config.yaml}.
\end{itemize}
As an example, the following shell command runs the code using 8 MPI processes and loads the configuration file \texttt{my\char`_config.yaml} from the path \texttt{/configs/}:

\mbox{\say{\texttt{mpiexec -np 8 python dga\char`_main.py -p /configs/ -c my\char`_config.yaml}}}.

Additionally, we feature extensive logging: every important step in the calculation will be logged. If it is started from a terminal, the logging will be done to the standard output. If the code is executed on a slurm-based cluster, one will find the logs in the job output file. The reason we employ a lot of logging is the ease of finding errors that might occur during a calculation.

\subsection{Configuration}

The configuration file is split into small blocks of configuration sections, which will be explained in detail in the following. For those entries where only a single data type is expected, the code will try to parse the input to this specific type. If the parsing was unsuccessful, a warning is logged and a default value for this variable is used. 

The first block of the configuration is concerned with the number of Matsubara frequencies one wants to use for the calculation:

\begin{minipage}{\textwidth}%
\begin{lstlisting}[language=yaml]
box_sizes:
  niw_core: 50  # int, default: -1
  niv_core: 30  # int, default: -1
  niv_shell: 20 # int, default:  0
\end{lstlisting}
\end{minipage}
The \say{core} region defines the frequency box used for explicitly solving the Bethe-Salpeter and Schwinger-Dyson equations, while the \say{shell} region sets the size of the asymptotic tails needed for vertex reconstruction through the method described in \secref{sec:vertex_asymptotics_urange}. It is possible to set the core frequencies to \say{-1}, which means that the number of Matsubara frequencies from the DMFT calculation will be taken.
\\\\
The next section is concerned with the Hamiltonian of the system, which currently is a little bit complex, since we wanted to adapt the existing configuration file from \texttt{DGApy} and added new features. It defines the lattice symmetries, the momentum grid sizes and the kinetic and interaction part of the Hamiltonian:

\begin{minipage}{\textwidth}%
\begin{lstlisting}[language=yaml]
lattice:
  symmetries: "two_dimensional_square"
                           (*@\hspace{-0.3em}@*) # string, default: "two_dimensional_square"
  type: "from_wannier90"    # string, default: "from_wannier90"
  hr_input: "/path/to/file" # string | list[float], default: None
  interaction_type: "local_from_dmft"
                           (*@\hspace{-0.3em}@*) # string, default: "local_from_dmft"
  interaction_input: "/path/to/file"
                           (*@\hspace{-0.3em}@*) # string | list[float], default: None
  nk: [16, 16, 1]           # list[int], default: [16, 16, 1]
  nq: [16, 16, 1]           # list[int], default: [16, 16, 1]
\end{lstlisting}
\end{minipage}
Here, we have a bunch of settings available that can be combined in different ways. First, we can define the lattice symmetry, where one has to enter a set of predefined symmetry sets. It is possible to extend the symmetries the code supports by adding the corresponding symmetry operations and symmetry sets in the code. The available symmetry sets are \texttt{two\char`_dimensional\char`_square}, \texttt{quasi\char`_one\char`_dimensional\char`_square}, \texttt{simultaneous\char`_x\char`_y\char`_inversion} and \texttt{quasi\char`_two\char`_dimensional\char`_square\char`_symmetries}. Next is the type of input to the kinetic part of the Hamiltonian we require. It can either be \texttt{from\char`_wannier90}, \texttt{from\char`_wannierHK} or \texttt{t\char`_tp\char`_tpp} (only for single-band input). The first input type is used if the subsequent field \texttt{hr\char`_input} references to a file, where the hopping elements are written in real space, whereas \texttt{from\char`_wannierHK} is used if the Hamiltonian in Fourier space is available as a file\footnote{Note that for the former, it is possible to use arbitrary $\vv{k}$-grid sizes, since the Fourier transform will be performed in the code, whereas for the latter one the correct grid size has to be entered.}. The last option is only used for single-band input, where the band distribution is that of a $t, t'$ and $t''$ hopping model only. If this type is specified, then a list of three float values in \texttt{hr\char`_input} is expected.

Since this code is capable of performing multi-orbital calculations with non-local interactions, we require the definition of a interaction type and an input. The available interaction types are \texttt{local\char`_from\char`_dmft}, \texttt{kanamori\char`_from\char`_dmft}, \texttt{kanamori} and \texttt{custom}. If \texttt{local\char`_from\char`_dmft} is specified, we expect a single-band input, where the interaction strength is taken from DMFT. Similar to that, the \texttt{kanamori\char`_from\char`_dmft} option takes the interaction values from DMFT and constructs the correct interaction matrix with the Kanamori-specific entries. In both cases, the field \texttt{interaction\char`_input} will not be read. If \texttt{kanamori} is specified, then we expect a list of four values in \texttt{interaction\char`_input}, which contains the number of orbitals as the first entry and then $U$, $J$ and $V$ as the next three entries. One can enter whole numbers for the number of bands and floating point values for the interaction strengths. Lastly, if one specifies \texttt{custom} as the interaction type, we then expect a path to a file, which is structured similarly to a real-space Hamiltonian file but with four orbital entries instead of just two. It is possible to include non-local interactions there. Furthermore, it is possible to set the sizes of both the $\vv{k}$- and $\vv{q}$-grid in the fields \texttt{nk} and \texttt{nq}, respectively. It is possible to not set \texttt{nq}, where \texttt{nk} will be used for the $\vv{q}$-grid. Note, that the execution of the Eliashberg equation is only possible if both the $\vv{k}$- and the $\vv{q}$-grid are of the same size.
\\\\
The next configuration section is concerned with the self-consistency calculation and defines the convergence behavior and mixing strategy of the self-consistency loop:

\begin{minipage}{\textwidth}%
\begin{lstlisting}[language=yaml]
self_consistency:
  max_iter: 20                      # int, default: 20
  save_iter: True                   # bool, default: True
  epsilon: 1e-6                     # float, default: 1e-4
  mixing: 0.3                       # float, default: 0.2
  mixing_strategy: "pulay"          # string, default: "linear"
  mixing_history_length: 2          # int, default: 3
  previous_sc_path: "path/to/files" # string, default: "./"
\end{lstlisting}
\end{minipage}
The field \texttt{max\char`_iter} is to set an upper boundary on the number of iterations during the self-consistency cycle. If the self-energy does not converge within \texttt{max\char`_iter} iterations, the iteration will stop. When setting \texttt{save\char`_iter} to \texttt{True}, the code will save the non-local $\dga$ self-energy for each iteration. The next parameter, \texttt{epsilon}, is set to determine the absolute value difference allowed for the last two calculated sigmas to be considered converged. The convergence will be tested on a predefined set of $\vv{k}$-points and frequencies $\nu$. Next, we can specify the \texttt{mixing} parameter that will be used in the mixing scheme and is a floating point number between $(0,1]$. We can furthermore specify the mixing scheme in the next parameter, called \texttt{mixing\char`_strategy}. Here, we can choose from either \texttt{linear} or \texttt{pulay} mixing, where if \texttt{pulay} is specified, it takes \texttt{mixing\char`_history\char`_length} number of previous self-energy results to construct a prediction. If the number of iterations is less than the history length, we will use linear mixing for these iterations and switch to Pulay mixing afterwards. The last parameter, \texttt{previous\char`_sc\char`_path} allows us to specify the path to a previous self-consistency iteration folder, which might not have been converged or which needs to be converged further or with different parameters. The program then takes the last iterations of the already performed calculation and uses these as a starting point for the next calculation. If the number of iterations from the previous calculation is equal or larger than \texttt{mixing\char`_history\char`_length} and \texttt{pulay} is specified, then Pulay mixing will be applied, otherwise we mix linearly.
\\\\
Given the improved speed and scalability of this code compared to \texttt{DGApy}, we have also implemented two $\lambda$-correction schemes which only work for single-band cases and can be configured with the following short section:

\begin{minipage}{\textwidth}%
\begin{lstlisting}[language=yaml]
lambda_correction:
  perform_lambda_correction: True # bool, default: False
  type: "spch"                    # string, default: "spch"
\end{lstlisting}
\end{minipage}
Here, setting \texttt{perform\char`_lambda\char`_correction} to \texttt{True} will perform a $\lambda$-correction if the input is a single-band model. Otherwise, an error is raised and the code exits. Notice that if one wants a one-shot lambda-corrected self-energy and calculation of the Eliashberg equation, then \texttt{max\char`_iter} and \texttt{mixing} in the \texttt{self\char`_consistency} section need to be set to \say{1}. The other parameter is the type of $\lambda$-correction which will be performed. Setting this to \texttt{spch} will renormalize both the density and magnetic susceptibilities, whereas setting it to \texttt{sp} will only renormalize the magnetic susceptibility.
\\\\
The next section is a very important one, since it is concerned with the location of the DMFT files and mainly defines their location and file names:

\begin{minipage}{\textwidth}%
\begin{lstlisting}[language=yaml]
dmft_input:
  type: "w2dyn"                # string, default: "w2dyn"
  input_path: "/path/to/files" # string, default: "./"
  fname_1p: "filename"         # string, default: "1p-data.hdf5"
  fname_2p: "filename"         # string, default: "g4iw_sym.hdf5"
  do_sym_v_vp: True            # bool, default: True
\end{lstlisting}
\end{minipage}
Currently, the \texttt{type} argument only supports \texttt{w2dyn} as input type, but this could be extended in the future to also support input files from other sources with other file structures. Next, \texttt{input\char`_path} needs to be set to the folder that contains the output data from \texttt{w2dynamics}. We require two files, one containing one-particle quantities (such as the DMFT Green's function and self-energy) and one containing two-particle quantities (such as the two-particle Green's function); their filenames can be set in \texttt{fname\char`_1p} and \texttt{fname\char`_1p}, respectively. Notice that \texttt{w2dynamics} outputs a large \texttt{Vertex.hdf5}-file, where we have to extract the relevant worm components beforehand. This can be done with the script \texttt{symmetrize.py}, which is also part of the repository. This will automatically extract all relevant worm components of the two-particle Green's function and writes it to a separate file containing the density and magnetic contributions. Lastly, we have the option to symmetrize the two-particle Green's function in $\nu$ and $\nu'$ before processing them further. This might be necessary sometimes to ensure numerical symmetry of the vertex functions.
\\\\
The second to last configuration section we will cover here is the output section which, as the name already suggests, covers the program's output settings, such as the output path and plotting instructions: 
 
\begin{minipage}{\textwidth}%
\begin{lstlisting}[language=yaml]
output:
  output_path: "/path"             # string, default: "./"
  do_plotting: True                # bool, default: "True"
  plotting_subfolder_name: "Plots" # string, default: "Plots"
  save_quantities: True            # bool, default: True
\end{lstlisting}
\end{minipage}
We are able to choose the path where the output of the program is saved to. If this field is chosen empty, the output will be put into a sub-folder inside the DMFT input folder. If we specify \texttt{do\char`_plotting}, then we plot a selective set of quantities and save them to the sub-folder specified in \texttt{plotting\char`_subfolder\char`_name}. We then have the possibility to additionally save almost all quantities that are calculated during the main program as \texttt{numpy} files, except for the pairing vertex and the full ladder-$\dga$ vertex. These are separately handled in the Eliashberg section, since they are only calculated if one wants to perform the calculation of the superconducting eigenvalue and gap function.
\\\\
Finally, the last configuration section is concerned with the Eliashberg equation and is configuring the eigenvalue and gap function-finding through a Lanczos method:

\begin{minipage}{\textwidth}%
\begin{lstlisting}[language=yaml]
eliashberg:
  perform_eliashberg: True     # bool, default: False
  save_pairing_vertex: False   # bool, default: False
  save_fq: False               # bool, default: False
  n_eig: 4                     # int, default: 4
  epsilon: 1e-9                # float, default: 1e-6
  symmetry: "random"           # string, default: "random"
  include_local_part: True     # bool, default: True
  subfolder_name: "Eliashberg" # string, default: "Eliashberg"
\end{lstlisting}
\end{minipage}
Here, the first setting is to let the program know if one wants to perform the Eliashberg equation to obtain the superconducting singlet and triplet eigenvalues and gap functions, respectively. If this setting is set to \say{False}, then the program will exit after the self-energy has been calculated. The next two settings allow to save the pairing vertex $\Gamma^{(\q=0)\k\kp}$ and the full ladder vertex $F^{\q\nu\nu'}$ to a file. These are set to \say{False} per default, as these objects can be quite large, especially the full vertex, which can reach up to hundreds of gigabytes. Both will be saved for the irreducible Brillouin zone only to save storage space. The following three settings, \texttt{n\char`_eig}, \texttt{epsilon} and \texttt{symmetry} are concerned with the power iteration that is performed to solve the Eliashberg equation. \texttt{n\char`_eig} allows the user to specify the number of largest eigenvalues and corresponding eigenvectors to be calculated using a Lanczos method. \texttt{epsilon} determines the target precision of the power iteration which is performed and \texttt{symmetry} determines the starting vector. There are a couple of options available for \texttt{symmetry} that may help speeding up the power iteration convergence if one knows the pairing symmetry of the gap function a priori: \texttt{random}, \texttt{p-wave-x}, \texttt{p-wave-y} and \texttt{d-wave}. In almost all cases it should suffice to pick a random starting vector for the power iteration. 

In principle one also needs the \textit{purely local} full and reducible vertices for the pairing vertex, however these can be omitted if one expects d-wave symmetry of the gap function, where the local vertex functions do not contribute to the eigenvalue and gap function due to symmetry. Setting \texttt{include\char`_local\char`_part} to \say{False} does exactly that. Not calculating the local part allows for a slightly larger frequency box and a better estimation of the eigenvalue and gap function for d-wave symmetry cases.

The last setting is the sub-folder name where the Eliashberg output will be saved to.
 
\subsection{Input files}

After the vertex calculation has been performed with \texttt{w2dynamics} \cite{Wallerberger2019}, we have to extract a couple of key quantities, such as the local self-energy, Green's function, interaction and two-particle Green's functions from the output files of \texttt{w2dynamics}. There is a script located in the repository, called \texttt{symmetrize.py}, which allows for the straightforward extraction of the two-particle Green's function from the vertex file of \texttt{w2dynamics}. A simple execution of this script yields a new file, where the density and magnetic components of $G^{\omega\nu\nu'}_{r;\mathfrak{1234}}$ are written to. This is handy, since the vertex file from \texttt{w2dynamics} is usually very large and it can be deleted after the symmetrization process. Then one is left with the \texttt{1p\char`-data.hdf5} and \texttt{g4iw\char`_sym.hdf5} files, which contain the one-particle and two-particle (symmetrized) quantities. These are, next to a file containing the real-space (e.g., \texttt{wannier\char`_hr.dat}) or Fourier-space (e.g., \texttt{wannier.hk}) Hamiltonian, the only input files needed for the execution of the $\dga$ algorithm. Notice that no file name is fixed and they can be specified in the corresponding \texttt{dga\char`_config.yaml} file.

\section{Numerical effort}

In the following, we will briefly analyze the numerical scaling of the implemented $\dga$ algorithm: For low temperatures, the calculation of the local vertex functions through the CT-HYB QMC solver is very often the most time-consuming part of a $\dga$ calculation. The numerical cost of evaluating the vertex scales as $\mathcal{O}(\beta^5 n_{o}^4)$ with a considerable prefactor due to the Monte Carlo sampling of \texttt{w2dynamics}. Here, $n_{o}$ is the number of orbitals (bands). This scaling arises because the local vertex contains $\mathcal{O}(n_{\omega}n_{\nu}^2n_{o}^4)$ components\footnote{In general, we assume that the number of Matsubara frequencies scales with $\beta$, i.e., $n_{\omega}\sim n_{\nu}\sim\beta$.}, while updating the CT-HYB hybridization matrix itself requires $\mathcal{O}(\beta^2)$ operations: the average expansion order and the matrix dimension both grow linearly with $\beta$.

To mitigate the cost, the asymptotic high-frequency form of the irreducible vertex, see \ref{sec:vertex_asymptotics_urange}, can be calculated with only two frequencies and therefore scales as $\mathcal{O}(\beta^4 n_{o}^4)$, allowing for a smaller frequency-box size in the QMC calculation. This makes calculations near superconducting transition temperatures feasible.

In the new code, the calculations are parallelized over the system's $\vv{q}$-grid and are diagonal in the bosonic Matsubara frequency $\omega$, yielding a numerical effort that scales with $\mathcal{O}(n_{\vv{q}}n_{\omega})$. For each $\vv{q}$ and $\omega$ entry, the dominant cost stems from operations using vectors and matrices whose dimensions are $N=n_{\nu}n_{o}^2$. Therefore, a straightforward inversion of a single matrix scales as $\mathcal{O}(N^3)$\footnote{In our implementation, we use the inversion provided by the \texttt{Python} package \texttt{scipy}, which employs the \texttt{LAPACK} functions \texttt{cgetrf}/\texttt{cgetri} or \texttt{zgetrf}/\texttt{zgetri}, depending on the precision required. These algorithms scale as $\mathcal{O}(N^3)$, whereas other algorithms with a lower complexity exist, such as Strassen-like or $N^{2.5}$-type inversions. However, these are typically only available in specialized libraries and are rarely used for dense numerical inversion in scientific computing because of stability and memory tradeoffs.}, whereas a matrix-vector multiplication scales with $\mathcal{O}(N^2)$. Consequently, the overall computational cost for these operations behaves as $\mathcal{O}(n_{\vv{q}}n_{\omega}n_{\nu}^3n_{o}^6)$ and $\mathcal{O}(n_{\vv{q}}n_{\omega}n_{\nu}^2n_{o}^4)$, respectively.

The evaluation of the self-energy through the Schwinger-Dyson equation has a different scaling, $\mathcal{O}(n_{\vv{k}}n_{\vv{q}}n_{\omega}n_{\nu}n_{o}^6)$ and only becomes dominant for very dense momentum grids.

\section{Result validation}

In order to check if the program's output is correct, we created a few simple test cases to validate our results:
\begin{enumerate}[label=(\roman*)]
\item First, we verify that the new code reproduces the same results as \texttt{DGApy}. While this comparison is limited to single-band models, it remains a crucial benchmark. Since the new implementation also supports $\lambda$-correction for single-band data, we compare both codes across various $\lambda$-correction schemes. Further details are provided in \secref{sec:validation_i}.
\item Next, we assess the correctness of the multi-orbital implementation. To this end, we constructed a fictitious two-orbital dataset in which both bands are identical and inter-orbital interactions are absent. Using the same parameters as in the previous test, we compare the results for each orbital with those from the single-band case. Aside from statistical noise, we expect the non-local self-energies and Green's functions to closely match. This validation strategy is discussed in \secref{sec:validation_ii}.
\item Finally, we test the invariance of the results under orbital-space rotations. Starting from the previous test case, we rotate the DMFT input data --- including the local one- and two-particle Green's functions, the DMFT self-energy, the Hamiltonian and the interaction matrix --- in orbital space. After performing the calculation, we apply the inverse rotation to the output and compare the result with the unrotated case. As detailed in \secref{sec:validation_iii}, we expect the two results to coincide within numerical precision.
\end{enumerate}
For the single-band input, we used $\beta=12.5$, a filling of $n=0.85$ and tight-binding parameters representative of the Ni-d$_{x^2-y^2}$ orbital in Nd$_{1-\text{x}}$Sr$_{\text{x}}$NiO$_{2}$ and related compounds: $t=1$, $t'=-0.25$ and $t''=0.12$, with an interaction strength of $U=8t$. Analogous parameters were used for the fictitious two-band input, where the total filling was set to $n=1.7$, corresponding to $0.85$ electrons per band. The momentum-space Hamiltonian was constructed to be orbital-diagonal, using identical hopping amplitudes for both bands. All input data were generated using the CT-HYB quantum Monte Carlo solver \texttt{w2dynamics}, run on the Vienna Scientific Cluster.

\subsection{Comparison to reference code with single-band model}\label{sec:validation_i}

We performed a full one-shot $\dga$ calculation using both \texttt{DGApy} and our new implementation, employing identical input files and parameters. Across all computed quantities, our implementation reproduced the results of the $\dga$ algorithm with floating-point accuracy. This agreement held for both the basic one-shot calculation without $\lambda$-corrections and for cases with $\lambda$-corrections applied --- whether in the spin channel alone or in both spin and charge channels. For instance, the absolute differences in the self-energies between the two implementations did not exceed $\mathcal{O}(10^{-11})$, while the five largest eigenvalues of the pairing vertex in the Eliashberg equation differed by at most $\mathcal{O}(10^{-6})$. This level of agreement was consistently achieved across different $\vv{k}$- and $\vv{q}$-grids and varying numbers of Matsubara frequencies $n_\omega$ and $n_\nu$.

\subsection{Disconnected two-band model}\label{sec:validation_ii}
\subsection{Rotated disconnected two-band model}\label{sec:validation_iii}

For this test, we use the fictitious disconnected two-band model introduced earlier. The goal is to verify that the orbitals remain properly separated and are not mixed in unintended ways. To this end, we implemented simple rotation routines in the code, which allow us to rotate the one- and two-particle vertex functions in orbital space. For the one-particle Green's function and self-energy this looks like
\begin{align}
	G_{\mathfrak{ab}}^{\k;\vartheta}=\left (S^{\vartheta}_{\mathfrak{1a}}\right )^{\!\dagger} G_{\mathfrak{12}}^{\k}\; S_{\mathfrak{2b}}^{\vartheta},
\end{align}
and similarly for $\Sigma_{\mathfrak{12}}^{\nu}$, whereas for any four-point vertex the rotation is performed like
\begin{align}
	F_{\mathfrak{abcd}}^{\q\k\kp;\vartheta}=\left( S^{\vartheta}_{\mathfrak{1a}}\right)^{\!\dagger} \left (S^{\vartheta}_{\mathfrak{2b}}\right)^{\!\dagger} F_{\mathfrak{1234}}^{\q\k\kp}\; S_{\mathfrak{3c}}^{\vartheta}S_{\mathfrak{4d}}^{\vartheta}.
\end{align}
Under the hood, this transforms the underlying creation and annihilation operators of the objects with $S$ being a two-dimensional rotation matrix, which simply looks like
\begin{align}
	S_{\mathfrak{12}}^{\vartheta} = \begin{pmatrix}
	\text{cos}(\vartheta) & \text{sin}(\vartheta)\\-\text{sin}(\vartheta)&\text{cos}(\vartheta)
	\end{pmatrix}_\mathfrak{12}
\end{align}
To perform this test, we applied a rotation by an angle $\vartheta$ to all input quantities from DMFT, including the kinetic part of the Hamiltonian and the interaction terms. The $\dga$ algorithm was then used to compute the self-energy, which we subsequently rotated back by $-\vartheta$. We compared the resulting self-energy to a reference calculation performed without any rotations and found them to agree up to floating-point precision, with $\sum_{\k}|\Sigma_{\text{rotated}}(\k)-\Sigma_{\text{not\;rotated}}(\k)|\lesssim 10^{-4}$. This test was repeated for various values of $\vartheta$ ranging from $-\pi$ to $\pi$, different numbers of Matsubara frequencies $n_\omega$ and $n_\nu$​, and varying $\vv{k}$-grid sizes --- all yielding indistinguishable results.

\end{document}
